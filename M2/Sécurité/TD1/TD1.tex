\documentclass[11pt,a4paper]{article}
\usepackage[utf8]{inputenc}
\usepackage[T1]{fontenc}
\usepackage[francais]{babel}
\usepackage{amsmath}
\usepackage{amsfonts}
\usepackage{amssymb}
\usepackage{graphicx}
\usepackage{fullpage}
\usepackage{tikz}
\usepackage{hyperref}

\title{Sécurité - TD 1}
\begin{document}
	
	\maketitle
	
	\section{Exercice 1 : paiement en ligne}
	
	Ce système est-il sûr ?\\
	Non.\\
	
	Sinon quelles vulnérabilités peut-on imaginer ?\\
	Le jeton n'étant composé que de 10 chiffres, une attaques par brute force est envisageable en moins de dix minutes car il n'y a que 10 Milliards de possibilités. De même, il n'est pas indiqué que la communication est chiffrée.
	
	\section{Exercice 2 : données médicales}
	
	Quels sont les objectifs de sécurité nécessaire pour cette base de données ?\\
	Disponibilité (nécessaire, suivre circuit d'un patient), intégrité (très important), confidentialité (obligatoire), traçabilité (authentification).\\
	
	\section{Exercice 3 : niveau de sécurité}
	Donner deux raisons qui expliquent l'augmentation constante de cette différence de niveau de sécurité.\\
	\begin{itemize}
		\item Réutilisation des même mots de passes d'une à plusieurs fois par jour.
		\item 
	\end{itemize}

	Décrire le genre de mesures permettant d'éviter ce phénomène.\\
	\begin{itemize}
		\item Changement régulier des mots de passes via les politiques de sécurité
		\item 
	\end{itemize}

	\section{Exercice 4 : analyse de risque}
	
	A partir de ces chiffres, calculer le risque annuel dû aux virus et aux défigurations et juger l'utilité de mettre en place les mesures de sécurité énoncées.\\
	

\end{document}