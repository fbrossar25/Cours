\documentclass[11pt,a4paper]{article}
\usepackage[utf8]{inputenc}
\usepackage[T1]{fontenc}
\usepackage[francais]{babel}
\usepackage{amsmath}
\usepackage{amsfonts}
\usepackage{amssymb}
\usepackage{graphicx}
\usepackage{fullpage}
\usepackage{tikz}
\usepackage{hyperref}

\title{Sécurité - CM}
\begin{document}
	
	\maketitle
	
	\section{Introduction}
	
	Adresse du prof : j2m@unistra.fr.\\
	
	\subsection{Historique}
	
	\begin{itemize}
		\item 1988, Ver morris, exploitation d'une faille de sendmail et de la commande finger sur UNIX ($\approx $ 6000 (10\%) machines contaminées).
		\item Travail pour résoudre le problème $\Rightarrow$ Création du CERT (Computer Emergency Response Team).
		\item Années 90 :
		\begin{itemize}
			\item Démocratisation d'internet
			\item Kevin Mitnick est le 1\ier{} hacker recherché par le FBI
			\item Premières exploitations de vulnérabilités par dépassement de tampon, injection SQL, XSS.
		\end{itemize}
		\item Début 2000 :
		\begin{itemize}
			\item Apparition de DDOS et des premiers botnets
			\item Premiers vers informatiques (I Love You, Code Red, ...)
		\end{itemize}
	\end{itemize}
	
	\subsection{Industrialisation}
	
	\begin{itemize}
		\item Cybercriminalité
		\begin{itemize}
			\item Apprition de gangs de hacker, notamment dans les pays de l'Est.
			\item Organisations criminelle sévissent sur Internet.
			\begin{itemize}
				\item Ex : Silk Road fermé en 2013 par le FBI (vente drogue et arme)
				\item Ex : rançon logiciels (ransomware)
			\end{itemize}
			\item Pertes estimés à plusieurs centaines de millions d'euros par an
			\begin{itemize}
				\item Agissent à partir de paradis numérique
			\end{itemize}
			\item Commerce d'armes d'intrusion ou de destruction numérique
			\begin{itemize}
				\item Ex : location de botnets, achat de 0-day
			\end{itemize}
		\end{itemize}
	\end{itemize}

\end{document}