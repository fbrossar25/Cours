\documentclass[11pt,a4paper]{article}
\usepackage[utf8]{inputenc}
\usepackage[T1]{fontenc}
\usepackage[francais]{babel}
\usepackage{amsmath}
\usepackage{amsfonts}
\usepackage{amssymb}
\usepackage{graphicx}
\usepackage{fullpage}
\usepackage{tikz}
\usepackage{hyperref}

\title{Sécurité - TD 3}
\begin{document}
	
	\maketitle
	
	\section{Exercice 1 : Principe de l'ISO 27001}
	
	\subsection{Quelles sont les différentes étapes d'une appréciation du risque ?}
	
	\begin{itemize}
		\item Identification des risques (vulnérabilités, conséquences, menaces, actifs)
		\item Estimation des risques (Estimation conséquences, probabilité d'occurrence, impacts)
		\item Évaluation du risque
	\end{itemize}

	\subsection{Question 2}
	
	\begin{itemize}
		\item Vulnérable car connecté au réseau interne et non mis à jour
		\item Toute personne malintentionnée ayant accès au réseau de l'entreprise + wifi ouvert au public s'il existe
		\item Pas de confidentialité, modification ou suppression de données (DIC)
		\item Installer les mises à jour de sécurité
		\item Régression de service dû aux mises à jour. Le serveur est peut être déjà infecté
		\item Il faut effectuer une analyse du serveur pour désinfection. Vérifier la compatibilité du matériel (Serveur de test).
	\end{itemize}

	\subsection{Question 3}

	\begin{itemize}
		\item Accès physiques
		\item Les utilisateurs
		\item Disponibilité, Intégrité, Confidentialité (DIC)
		\item Mettre des porte et des serrures
		\item Mauvaises gestion des mesures de sécurité misent en place
		\item Durcir l'accès (alarme si la porte reste ouverte, etc...)
	\end{itemize}

	\subsection{Question 4}
	
	\begin{itemize}
		\item Accepter
		\item Réduire (prendre des mesures)
		\item Refuser (supprimer le service)
		\item Transférer (sous-traiter)
	\end{itemize}

	\subsection{Question 5}
	
	Oui, si le risque à un impact faible et une probabilité d'occurrence faible, il peut être définis comme acceptable.
	
	\subsection{Question 6}
	
	Malgré les mesures de sécurité misent en place, certaines peuvent être réduites mais pas évitées totalement.
	
	\subsection{Question 7}
	
	Elle doit donner son approbation pour accepter ou refuser le risque résiduel.
	
	\subsection{Question 8}
	
	\begin{itemize}
		\item Objectifs de sécurité sélectionnés
		\item Mesures de sécurité retenues avec justification
		\item Mesures de sécurité effectivement mises en place
		\item Mesures de sécurité non retenues avec justification
	\end{itemize}

	\subsection{Question 9}
	
	Elle sert à vérifier que l'on a rien oublié.
	
	\subsection{Question 10}
	
	Simplification de l'audit.
	
	\subsection{Question 11}
	
	Voir chapitre 7.2 + 4.2.3b du document de la certification ISO 27001.
	
	\subsection{Question 12}
	
	Permet de certifier un SMSI.
	
\end{document}