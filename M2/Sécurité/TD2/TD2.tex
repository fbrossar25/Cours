\documentclass[11pt,a4paper]{article}
\usepackage[utf8]{inputenc}
\usepackage[T1]{fontenc}
\usepackage[francais]{babel}
\usepackage{amsmath}
\usepackage{amsfonts}
\usepackage{amssymb}
\usepackage{graphicx}
\usepackage{fullpage}
\usepackage{tikz}
\usepackage{hyperref}

\title{Sécurité - TD 2}
\begin{document}
	
	\maketitle
	
	\section{Exercice 1 : Chiffrement symétrique vs asymétrique}
	
	\begin{enumerate}
		\item $\sum_{i=1}^{n-1} = \frac{(n-1)(n-2)}{2}$ avec $n = 7$, il faut donc 21 clés symétriques.
		\item AES car il est un standard fiable.
		\item 7, une clé privé et une publique par personne.
		\item Sa clé privé et la clé publique d'Alice.
		\item RSA
		\item Chiffrement plus rapide, possibilité de signature par l'expéditeur
	\end{enumerate}

	\section{Exercice 2 : Clé privée}
	
	\begin{enumerate}
		\item Non il ne peut plus envoyer de courriers électroniques. Il peut en recevoir car il à toujours sa clé publique.
		\item Il ne peut pas signer ses courriers mais peut vérifier la signatures de ceux qu'il reçoit.
		\item Si le message est claire il peut utiliser la clé publique pour retrouver la clé privée. Il doit créer un nouveau couple de clés privée/publique.
	\end{enumerate}

	\section{Exercice 3 : Certificats}
	
	\begin{enumerate}
		\item Fichier contenant des informations tel que l'identité du signataire, l'algorithme de chiffrement, la clé publique du titulaire, etc...
		\item La différence majeure est que les certificats PGP n'ont pas d'autorités de certifications.
		\item Vérification de la présence du certificat dans le magasins de certificats du navigateur.
	\end{enumerate}

	\section{Exercice 4 : Certificats}
	
	\begin{enumerate}
		\item Le signataire des deux certificats est le même.
		\item Génération d'une même clé publique possible.
		\item Les deux certificats peuvent déchiffrer les messages de l'autre.
	\end{enumerate}

	\section{Exercice 5 : Attention connexion non certifiée}
	
	\begin{itemize}
		\item Date certificat expirée
		\item Autorité de certificat non officielle
		\item Autorité de certificat non présente dans le magasin
		\item Erreur dans le certificat
		\item Horloge système déreglée
		\item Certificat révoqué
		\item Protocole utilisé par la certif obsolète
	\end{itemize}
	
	\section{Exercice 6 : Diffie-Hellman}
	
	\section{Exercice 7 : RSA}
	
	\begin{align*}
		(n,e) &= (33, 3)\\
		n &= 33 = 3 * 11 \text{ (Nombre premiers)}\\
		p &= 3\\
		q &= 11\\
		e * d &= 1\ \%\ (p-1)(q-1)\\
		e * d &= 1\ \%\ 20\\
		3 * d - 1 &= 20\\
		\frac{d}{3}&= \frac{21}{3}\\
		d &= 7\\
		\text{Clé privée (n,d) } &= (33, 7)
	\end{align*}

	\section{Exercice 8 : RSA}
	
	\begin{enumerate}
		\item 
		\item 
		\item 
	\end{enumerate}

\end{document}