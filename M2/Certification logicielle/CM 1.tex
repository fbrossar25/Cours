\documentclass[11pt,a4paper]{article}
\usepackage[utf8]{inputenc}
\usepackage[T1]{fontenc}
\usepackage[francais]{babel}
\usepackage{amsmath}
\usepackage{amsfonts}
\usepackage{amssymb}
\usepackage{graphicx}
\usepackage{fullpage}
\usepackage{tikz}
\usepackage{hyperref}

\title{Certification logicielle - CM}
\begin{document}
	
	\maketitle
	
	\section{Crise du logiciel}
	
	\begin{itemize}
		\item Constat
			\begin{itemize}
				\item Délais non-tenus
				\item budget dépassé
				\item qualité médiocre
			\end{itemize}
		\item Il faut passer de l'artisanat au stade industriel
			\begin{itemize}
				\item formation des "informaticiens"
				\item concevoir des méthodes et des outils de développement
			\end{itemize}
	\end{itemize}
	
	\section{Spécifier, c'est difficile}
	
	\begin{itemize}
		\item Spécifications informelles :
			\begin{itemize}
				\item cahier des charges qui doit être précis, non ambigü, non redondant...
				\item prévoir plein de scénarios pour que la correction ne devienne pas de la robustesse
				\item est-ce que le logiciel répond au cahier des charges ?
			\end{itemize}
		\item Spécifications formelles, plusieurs approches :
			\begin{itemize}
				\item logique
				\item algébrique (logique équationnelle avec conditions)
				\item basées sur le modèle
				\item on peut (essayer de) prouver que le programme répond bien à sa spécif.
			\end{itemize}
	\end{itemize}

	\section{Certification 2}
	\subsection{Reprise de la logique de Hoare}
	Rappel : Logique de Hoare $\Rightarrow$ mélange un langage de programmation et des spécifications logiques
	
	Langage ? : \begin{itemize}
		\item langage des tant que (jouet - Turing complet)
		\item langages courants (C, JAVA, PHP, ...)
	\end{itemize}

	Définition du langage ;
		\begin{itemize}
			\item syntaxique - grammaire
			\item sémantique 	\begin{itemize}
									\item de style sémantique dénotationnelle (mathématique pas trop formel)
									\item intuitive
									\item opérationnelles (vu tout à l'heure)
								\end{itemize}
		\end{itemize}
	
	Syntaxe : \begin{itemize}
		\item id ou variable
		\item nombres entiers, flottants, ....
		\item booléens
	\end{itemize}

	Expressions : 2 types (arithmétiques, booléennes)
	
	\begin{align*}
		E_a \rightarrow I_d |&\\
							Z |&\\
							-E_a |&\\
							E_a + E_a |& E_a - E_a | E_a * E_a | E_a / E_a\\
	\end{align*}
	
	Instructions :
	\begin{align*}
		I \rightarrow skip\ |&\\
		I; I |&\\
		if\ E_b\ then\ I\ else\ I\ |&\\
		I_d := E_a |&\\
		while\ E_b\ do\ I&\\
	\end{align*}
	
	Exemple :
	\begin{align*}
		x &:= 3\\
		x &:= x + 1\\
		x &:= 0; x:= x + 1\\
		if\ x < y\ &then\ m := x\ else\ m := y\\
	\end{align*}
	
	En principe, pour des langages jouets, la sémantique instruite est non ambiguë.
	Sinon c'est réglé par le manuel de référence ou par des normes.
	
\end{document}