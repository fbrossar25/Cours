\documentclass[11pt,a4paper]{article}
\usepackage[utf8]{inputenc}
\usepackage[T1]{fontenc}
\usepackage[francais]{babel}
\usepackage[a4paper, bottom=3.5em, top=3em]{geometry}
\usepackage{amsmath}
\usepackage{amsfonts}
\usepackage{mathtools}
\usepackage{amssymb}
\usepackage{graphicx}
\usepackage{tikz}
\usepackage[hidelinks]{hyperref}

\title{Bioinformatique - Cours}
\begin{document}
	
	\maketitle
	\section{Biologie générale}
	Les archés sont extremophiles.
	
	
	\section{Biologie moléculaire}
	Alphabet ADN = \{A,C,G,T\} et ARN = \{A,C,G,U\}.\\
	Base + Ribos ou désoxyribose = nucléoside.\\
	Nucléoside + acide phosphorique = mononucléotide.\\
	Ensemble de mononucléotide = Brin d'ADN.\\
	
	\section{Codes}
	\begin{itemize}
		\item 64 codons (AAA, AAC, ..., TTG, TTT)
		\item 20 acides aminés
	\end{itemize}
	Plusieurs possibilit"d mathématiques
	\begin{itemize}
		\item Diamond Code
		\item Triangle Code
		\item Comma-Free Code
		\item Code Circulaire
	\end{itemize}
	La solution n'est pas un code mathématique.\\
	$$ADN \xrightarrow{transcription} ARN \xrightarrow{traduction} Protéine$$
	\begin{itemize}
		\item Application surjective
		\item 1 codon $\rightarrow$ 1 AA
	\end{itemize}
\end{document}