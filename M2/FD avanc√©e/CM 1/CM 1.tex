\documentclass[11pt,a4paper]{article}
\usepackage[utf8]{inputenc}
\usepackage[T1]{fontenc}
\usepackage[francais]{babel}
\usepackage{amsmath}
\usepackage{amsfonts}
\usepackage{amssymb}
\usepackage{graphicx}
\usepackage{fullpage}
\usepackage{tikz}
\usepackage{hyperref}

\title{Fouille de données avancée - CM1}
\begin{document}
	
	\maketitle
	
	\section{Intro / définitions}
	
	Types de satellite : optique et radar sont les plus courants. Lidar (ppur l'altitude), et d'autres existent.
	
	\subsection{Résolution spatiale}
	
	Correspond à la surface terrestre englobée dans un pixel
	\begin{itemize}
		\item Basse résolution : quelques centaines de mètres par pixel
		\item haute résolution : 5 mètres par pixel
		\item Très haute résolution : 0,6 mètres par pixel
	\end{itemize}

	Pour l'analyse d'image, la résolution nécessaire n'est pas la même, par exemple isoler les zones industrialisées nécessite que de la haute résolution (5-10m/px), pas plus.\\
	$\Rightarrow$ Différents niveaux d'analyse
	
	Exemple du glissements de terrain : Basse réso pour générale, haute réso pour différencier les partie du glissement de terrain, très haute réso pour trouver les fissures.
	
	\subsection{Résolution spectrale}
	
	Pourquoi ne pas se limiter au visible ?\\
	
	\begin{itemize}
		\item Radiométrie des objets terrestres
		\item Problème de la très haute résolution : on vois les ombres des bâtiments apparaitre
		\item Bandes hyperspectrale : Beaucoup plus de bande que d'habitude $\Rightarrow$ permet par exemple de pouvoir isoler des espèces dans la végétation.
		\item Satellites multispectraux : Entre 3 et 10 bandes
	\end{itemize}

	\subsection{Résolution temporelle}
	
	C'est la fréquence d'acquisition d'image d'une même zone. Aussi nommé la fréquence de revisite d'une satellite.\\
	
	Délai de revisite de plusieurs mois à quelques semaine voire quelques jours $\Rightarrow$ plusieurs tour de la Terre pour photographier une zone.
	
	\newpage
	
	\subsection{Quelques problèmes}
	
	\paragraph{Correction d'images}
	\begin{itemize}
		\item Le bleu est consommé à l'aller et au retour (atmosphère)
		\item Angle de du soleil par rapport au satellite, il ne passe pas toujours au zénith. Les immeuble sont par exemple plus grand à cause de l'angle de prise de vue.
		\item Absorption, ombres, distorsion géométrique, diffraction
		\item Objet qui émette de la lumière
		\item Turbulence (air qui déforme la lumière, ...)
	\end{itemize}
	
	\paragraph{Pixels vs objets}
	\begin{itemize}
		\item Segmentation : regroupement de pixels suivant un critère d'homogéneité
	\end{itemize}
	\paragraph{Quels traitements ?}
	\begin{itemize}
		\item Extraction d'objets (recherche d'objets géographiques cible : bâtiments, champs, ...)
		\item Classification de l'image
			\begin{itemize}
			\item Approche supervisée : apprendre le modèle de classification des pixels ou des objets
			\item Approche non supervisée
			\end{itemize}
		\item Classification supervisée
		\item Nombres de classes
	\end{itemize}

	\paragraph{Et l'expert ?}
	\begin{itemize}
		\item Peu disponible (ou cher...)
		\item Peu commettre des erreurs
		\item Doit être guidé afin de fournir des connaissances utiles, compréhensibles et opérables
		\item Formalisation de ses connaissances : Exemple $\Rightarrow$ Ontologie
		\item Domaines transfert : Échelle d'analyse, zone, satellite, date, ...
	\end{itemize}

	\subsection{Challenges}
	\begin{itemize}
		\item Utiliser la complémentarité des méthodes existantes
		\item Valoriser les informations temporelles
		\item Intégrer l'expert
		\begin{itemize}
			\item Apprentissage (segmentation / classification) actif
			\item ???
		\end{itemize}
	\end{itemize}
	
\end{document}