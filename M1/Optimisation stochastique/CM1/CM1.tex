\documentclass[11pt,a4paper]{article}
\usepackage[utf8]{inputenc}
\usepackage[T1]{fontenc}
\usepackage[francais]{babel}
\usepackage{amsmath}
\usepackage{amsfonts}
\usepackage{amssymb}
\usepackage{graphicx}
\usepackage{fullpage}
\usepackage{tikz}
\usepackage{hyperref}

\title{Optimisation stochastique - CM1}
\begin{document}
	
	\maketitle
	
	\section{Problématique}
	
	\subsection{Problèmes difficiles}
	
	\begin{itemize}
		\item \textit{Problèmes inverses :} solution existante d'un problème dans un sens, mais pas dans l'autre
		\item Problèmes NP-Complet très difficiles à résoudre
		\item Méthode Monte Carlo (recherche aléatoire)
		\item utiliser la parallèlisation afin d'obtenir un comportement émergent (utiliser plusieurs machine qui s'entraident)
		\item \textit{Système complexe :} Système composé d'un grand nombre d'entités autonomes en interaction créant plusieurs niveaux d'organisation collective (multi-échelles) aboutissant à des comportements émergents/immergents
	\end{itemize}
	\vspace{1em}
	Exemples de systèmes complexes :
	\begin{itemize}
		\item Forme des dunes (grains de sables irréguliers)
		\item Animaux (chaques cellules est autonomes)
		\item Insectes (colonies, fourmilières)
	\end{itemize}
	
	
	\subsection{Types de problèmes}
	
	
	
\end{document}