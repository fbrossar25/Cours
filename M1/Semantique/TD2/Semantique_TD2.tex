\documentclass[11pt,a4paper]{article}
\usepackage[utf8]{inputenc}
\usepackage[T1]{fontenc}
\usepackage[francais]{babel}
\usepackage{amsmath}
\usepackage{amsfonts}
\usepackage{amssymb}
\usepackage{graphicx}

\title{Sémantique TD2}

\newcommand{\ra}{\rightarrow}

\begin{document}
	\section{Algèbres, congruences et modèles}
	Voir notes écrites
	\section{Spécifications}
	\subsection{Exercice 1}
	[Nat Bool]\\
	
	op zero : -> Nat\\
	op succ : Nat -> Nat\\
	ops plus, mult : Nat Nat -> Nat\\
	eq plus(zero, m) = m\\
	eq plus (succ(n), p) = succ (plus n p)\\
	eq mult (zero,n) = zero\\
	eq mult (succ(p),q) = plus (q, mult(p,q))\\
	\begin{equation*}
	\text{eq somme(x,y)} =\begin{cases}
	somme(y,x), & \text{si $y<x$}\\
	x, & \text{si $x=y$}\\
	x + somme(x+1,y) & \text{sinon}
	\end{cases}
	\end{equation*}
	eq egal(zero, zero) = true\\
	eq egal(succ(n), zero) = false\\
	eq egal(zero, succ(n)) = false\\
	egal(succ(p),succ(q)) = egal(p,q)\\
	\newpage
	\section{Logique équationnelle}
	\subsection{Exercice 1}
	\begin{align*}
	\begin{cases}
	\text{$M + 0 = M$ (1)}\\
	\text{$M + s(N) = s(M + N)$ (2)}
	\end{cases}\\
	s(s(0)) + (s(0) + s(s(s(0)))) &=^{(2)} s(s(0)) + (s(s(0) + s(s(0))))\\
	&=^{(2)} s(s(s(0))) + (s(0) + s(s(0))\\
	&=^{(2)} s(s(s(0))) + (s(s(0) + s(0)))\\
	&=^{(2)} s(s(s(s(0)))) + (s(0) + s(0))\\
	&=^{(2)} s(s(s(s(0)))) + (s(s(0) + 0))\\
	&=^{(2)} s(s(s(s(0)))) + (s(s(0)))\\
	&=^{(1)} s(s(s(s(s(0)) + s(0)))\\
	&=^{(2)} s(s(s(s(s(s(0)))))) + 0\\
	&= s(s(s(s(s(s(0))))))
	\end{align*}
	\newpage
	\subsection{Exercice 2}
	On dispose des axiomes suivants :
	\begin{enumerate}
		\item $(X + Y) + Z = X + (Y + Z)$
		\item $e + X = X$
		\item $(-X) + X = e$
	\end{enumerate}
	Montrer que $\forall X, X + (-X) = e$ est un théorème déductif.\\
	Astuce :\\
	\begin{align*}
		X + (-X) &= --X + -X +X\\
		&= --X + (-X + X)\\
		&= --X + e\\
		\text{et } X + (-X) &= --X + -X +X\\
		&= e + X\\
		&= X\\
		\text{J'en déduis que } &X = --X + e\\ \\
		X + (-X) &= (--X + e) + (-X)\\
		&= --X + (e + -X)\\
		&= --X + -X\\
		&= e\\
	\end{align*}
	\newpage
	\section{Logique inductive}
	\subsection{Exercice 1}
	Voir les note d'adèle
	
	\subsection{Exercice 2}
	On dispose des axiomes suivants :
	\begin{enumerate}
		\item $0 + M = M (A)$
		\item $s(N) + M = s(N + M) (B)$
	\end{enumerate}
	Montrons que $X + Y = Y + X$\\
	Par induction
	\begin{align*}
		\text{(1) } 0 + Y &= Y + 0\\
		\text{(2) } N + Y = Y + N &\rightarrow s(N) + Y = Y + s(N)\\ \\
		\text{1) }
		\left.
		\begin{array}{rcr}
		0 + Y &=^{(A)} Y\\
		Y + 0 &=^{(1)} Y\\
		\end{array}
		\right\}
		\text{d'où } 0 + Y &= Y + 0\\ \\
		\text{2) }
		\left\{
		\begin{array}{rcr}
		s(N) + Y &=^{(B)} s(N +Y)\\
		Y + s(N) &=^{(2)} s(Y + N)\\
		\end{array}
		\right.
	\end{align*}
	Par hypothèse d'induction
	
\end{document}