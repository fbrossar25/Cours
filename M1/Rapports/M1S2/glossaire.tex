%\newglossaryentry{SampleGlossRef} 
%{
%    name={Sample Glossary Reference},
%    description={Sample glossary reference description},
%    text={Sample glossary reference} %\gls{SampleGlossRef}
%    plural={Sample Glossary References} %\glspl{SampleGlossRef}
%}

%\newacronym[longplural={Frames per Second}]{fpsLabel}{FPS}{Frame per Second}
%\newacronym{lvm}{LVM}{Logical Volume Manager}

\newacronym{ufr}{UFR}{Unité de Formation de Recherche}

\newacronym{ilc}{ILC}{Ingénierie des Logicielles et des Connaissances}

\newacronym{ihm}{IHM}{Interface Homme-Machine}

\newacronym{es}{ÉS}{Électricité de Strasbourg}

\newglossaryentry{retroingenierie}
{
    type=\acronymtype,
    name={Rétro-ingénierie},
    description={La rétro-ingénierie, ou ingénierie inverse, est l'activité qui consiste à étudier un objet pour en déterminer le fonctionnement interne ou la méthode de fabrication.\cite{wiki:RetroIngenierie}},
    text={rétro-ingénierie}
}

\newglossaryentry{rest}
{
    name={REST},
    description={REST (representational state transfer) est un style d'architecture logicielle, notamment utilisé dans les applications web\cite{wiki:REST}},
    first={REST (representational state transfer)},
    text={Representational state transfer}
}

\newglossaryentry{dmz}
{
    type=\acronymtype,
    name={DMZ},
    description={En informatique, une zone démilitarisée (ou DMZ, de l'anglais demilitarized zone) est un sous-réseau séparé et isolé du réseau local par un pare-feu. Ce sous-réseau contient les machines étant susceptibles d'être accédées depuis Internet.\cite{wiki:DMZ}},
    text={DMZ}
}

\newglossaryentry{tma}
{
    type=\acronymtype,
    name={TMA},
    description={La Tierce Maintenance Applicative consiste à externaliser la maintenance de toutes ou d'une partie des applications d'une entreprise auprès d'un prestataire},
    first={tierce maintenance applicative (TMA)},
    text={TMA}
}

\newglossaryentry{gwt}
{
    name={GWT},
    description={Framework permettant de créer et maintenir des applications web dynamiques mettant en œuvre JavaScript, en utilisant le langage et les outils Java},
    first={GWT (Google Web ToolKit)},
    text={GWT}
}

\newglossaryentry{gxt}
{
    name={GXT},
    description={Framework permettant de créer des applications web riches en utilisant GWT},
    first={GXT},
    text={GXT}
}

\newglossaryentry{vm}
{
    name={Machine virtuelle},
    description={Une machine virtuelle est une simulation d'un appareil informatique créée par un logiciel},
    first={machines virtuelles},
    text={Machine virtuelle}
}

\newglossaryentry{sts}
{
    type=\acronymtype,
    name={STS},
    description={Sprinsource Tool Suite est une distribution du logiciel \gls{eclipse} facilitant le développement d'application Web JAVA utilisant le framework \gls{spring}},
    text={STS}
}

\newglossaryentry{eclipse}
{
    name={Eclipse},
    description={Eclipse est un \gls{ide} libre et extensible pour le language JAVA},
    text={Eclipse}
}

\newglossaryentry{ide}
{
    type=\acronymtype,
    name={IDE},
    description={. Un IDE (pour Integrated Development Environment) rassemble un enemble d'outils permettant d'augmenter la productivité des développeurs en un seul logiciel},
    text={IDE}
}

\newglossaryentry{spring}
{
    name={Spring},
    description={Le framework Spring permet de construire et de définir l'infrastructure d'une application JAVA},
    text={Spring}
}

\newglossaryentry{maven}
{
    name={Maven},
    description={Outil de gestion et d'automatisation de production des projets logiciels Java},
    text={Maven}
}

\newglossaryentry{svn}
{
    name={SVN},
    description={Subversion (en abrégé svn) est un logiciel de gestion de versions\cite{wiki:svn}},
    text={SVN}
}


\newglossaryentry{webservice}
{
    name={Webservice},
    description={Logiciel sans interface, accessible depuis une URL, permettant la communication et l'échange de données entre applications et systèmes hétérogènes},
    text={Webservice}
}

\newglossaryentry{rpc}
{
    type=\acronymtype,
    name={RPC},
    description={En informatique, Remote Procedure Call (RPC) désigne un protocole réseau permettant de faire appels à des procédures sur un ordinateur distants à l'aide d'un serveur d'applications},
    text={RPC}
}

\newglossaryentry{dao}
{
    type=\acronymtype,
    name={DAO},
    description={Un DAO (Data Access Object) est un patron de conception permettant d'ajouter une couche d'abstraction concernant le stockage de données},
    text={DAO}
}

\newglossaryentry{dto}
{
    type=\acronymtype,
    name={DTO},
    description={Un objet de transfert de données (Data Transfer Object en anglais) est un patron de conception utilisé dans les architectures logicielles objet. Il permet de simplifier la transmission de donner entre les différentes couches d'une application},
    text={DTO}
}

\newglossaryentry{ajax}
{
    type=\acronymtype,
    name={AJAX},
    description={L'architecture AJAX, acronyme d'Asynchronous JavaScript and XML, permet de créer des applications web dynamique et interactive\cite{wiki:AJAX}. Cela permet entre autre de modifer une partie de la page sans devoir la recharger entièrement},
    text={AJAX}
}

\newglossaryentry{sgbd}
{
    type=\acronymtype,
    name={SGBD},
    description={En informatique, un système de gestion de base de données (abr. SGBD) est un logiciel système destiné à stocker et à partager des informations dans une base de données\cite{wiki:SGBD}},
    first = {Système de Gestion de Base de Données (SGBD)},
    text={SGBD}
}

\newglossaryentry{hibernate}
{
    name={Hibernate},
    description={Hibernate est un framework open source gérant la persistance des objets en base de données relationnelle\cite{wiki:hibernate}},
    text={Hibernate}
}