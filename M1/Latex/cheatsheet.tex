\documentclass[11pt,a4paper]{article}
\usepackage[utf8]{inputenc}
\usepackage[T1]{fontenc}
\usepackage[francais]{babel}
\usepackage{amsmath}
\usepackage{amsfonts}
\usepackage{amssymb}
\usepackage{graphicx}
\usepackage{stmaryrd}
\usepackage{tabu}
\usepackage{array}
\usepackage{verbatim}
\usepackage{listings}
\usepackage{color}


\title{\LaTeX cheatsheet}

\newcommand{\q}[1]{\begin{verbatim}#1\end{verbatim}}
\newcommand{\row}[1]{#1\\\hline}
\newcommand{\bs}{\backslash}
\def\X#1{$#1$ &\tt\string#1} %cite une commande
\def\W#1#2{$#1{#2}$ &\tt\string#1\string{#2\string}}
\newcolumntype{C}{>{$}c<{$}}

\begin{document}
	\maketitle
	
	\section{Symboles}
	
	\subsection{Lettres grèques}
	\begin{center}
		\begin{tabu}{|c|c|c|}
			\everyrow{\hline}
			\hline
			\text{Symbole} & Commande & Description\\
			\X\lambda &\\
			\X\Pi &\\
			\X\Theta &\\
		\end{tabu}
	\end{center}
	
	\subsection{Logique}
	
	\begin{center}
		\begin{tabu}{|c|c|c|}
			\everyrow{\hline}
			\hline
			\text{Symbole} & Commande & Description\\
			\X\Leftrightarrow & Équivalent\\
			\X\top & True\\
			\X\bot & False\\
			\X\lor & Disjonction\\
			\X\land & Conjonction\\
			\X\implies & Implique\\
		\end{tabu}
	\end{center}
	
	\subsection{Ensemble}
	
	\begin{center}
		\begin{tabu}{|c|c|c|}
			\everyrow{\hline}
			\hline
			\text{Symbole} & Commande & Description\\
			\X\cup & Union\\
			\X\cap & Intersection\\
			\X\subseteq & Sous-ensemble\\
			\X\supseteq & Sur-ensemble\\
			\X\nsubseteq & Non sous-ensemble\\
			\X\nsupseteq & Non sous-ensemble\\
			\X\nsupseteq & Non sur-ensemble\\
			\X\in & Appartient\\
			$\mathbb{N}$ & \tt\string\mathbb\{N\} & Ensemble des entiers naturels\\
		\end{tabu}
	\end{center}
	
	\subsection{Arithmétique}
	
	\begin{center}
		\begin{tabu}{|c|c|c|}
			\everyrow{\hline}
			\hline
			\text{Symbole} & Commande & Description\\
			\X\sqsubseteq & Est plus définie que\\
			\X\exists & Il existe\\
			\X\forall & Pour tout\\
			$\sum_{i=0}^{n}$ & \tt\string\sum\_\{i=0\}\string^\{n\} &\\
			\X\to & Vers\\
			\X\equiv & Est identique à\\
			\X\neg & Negation\\
			\X\nless  & Non inférieur\\
			\X\ngtr & Non supérieur\\
			\X\ngeqslant & Non supérieur ou égal\\
			\X\nleqslant & Non inférieur ou égal\\
			$\llbracket \ \ \rrbracket$ & \tt\string\llbracket \tt\string\rrbracket & Intervalle d'entiers\\
			\X\int & Intégrale\\
		\end{tabu}
	\end{center}

	\subsection{Autres}
	{%arraystretch redefined in this block only
	\renewcommand{\arraystretch}{1.2}
	\begin{center}
		\begin{tabu}{|c|c|c|}
			\everyrow{\hline}
			\hline
			\text{Symbole} & Commande & Description\\
			\W\overline{abc} &\\
			\W\sqrt{abc} &\\
			$\sqrt[n]{abc}$&\tt\string\sqrt[n]\{abc\}&\\
			$\frac{abc}{xyz}$&\tt\string\frac\{abc\}\{xyz\}&\\
		\end{tabu}
	\end{center}
	}

	\section{Commandes}
	
	\subsection{Longues formules mathématiques}
	\begin{verbatim}
		\begin{eqnarray*}
		\left(1+x\right)ˆn & = & 1 + nx + \frac{n\left(n-1\right)}{2!}xˆ2 \\
		& & {} + \frac{n\left(n-1\right)\left(n-2\right)}{3!}xˆ3 \\
		& & {} + \frac{n\left(n-1\right)\left(n-2\right)\left(n-3\right)}{4!}xˆ4 \\
		& & {} + \ldots
		\end{eqnarray*}

	\end{verbatim}
	\begin{eqnarray*}
		\left(1+x\right)ˆn & = & 1 + nx + \frac{n\left(n-1\right)}{2!}xˆ2 \\
		& & {} + \frac{n\left(n-1\right)\left(n-2\right)}{3!}xˆ3 \\
		& & {} + \frac{n\left(n-1\right)\left(n-2\right)\left(n-3\right)}{4!}xˆ4 \\
		& & {} + \ldots
	\end{eqnarray*}

	\newpage
	
	\begin{align*}
	P^{\prime}_{B}(t) &= \lim\limits_{T \to 0}\left(\frac{P_{B}(t + T) - P_{B}(t)}{T}\right)\\
	&= \lim\limits_{T \to 0}\left(\frac{\alpha_{T}(1 - P_{B}(t)) - (n - 1)\alpha_{T}P_{B}(t)}{T}\right)\\
	&= (1 - P_{B}(t)) \lim\limits_{T \to 0}\left(\frac{\alpha_{T}}{T}\right) - (n - 1) P_{B}(t)\lim\limits_{T \to 0}\left(\frac{\alpha_{T}}{T}\right)\\
	\end{align*}
	
	
	\definecolor{mygreen}{rgb}{0,0.6,0}
	\definecolor{mygray}{rgb}{0.5,0.5,0.5}
	\definecolor{mymauve}{rgb}{0.58,0,0.82}
	\subsection{Code source}
	\lstset{ 
		backgroundcolor=\color{white},   % choose the background color; you must add \usepackage{color} or \usepackage{xcolor}; should come as last argument
		basicstyle=\footnotesize,        % the size of the fonts that are used for the code
		breakatwhitespace=false,         % sets if automatic breaks should only happen at whitespace
		breaklines=true,                 % sets automatic line breaking
		captionpos=b,                    % sets the caption-position to bottom
		commentstyle=\color{mygreen},    % comment style
		deletekeywords={...},            % if you want to delete keywords from the given language
		escapeinside={\%*}{*)},          % if you want to add LaTeX within your code
		extendedchars=true,              % lets you use non-ASCII characters; for 8-bits encodings only, does not work with UTF-8
		frame=single,	                   % adds a frame around the code
		keepspaces=true,                 % keeps spaces in text, useful for keeping indentation of code (possibly needs columns=flexible)
		keywordstyle=\color{blue},       % keyword style
		language=caml,                 % the language of the code
		morekeywords={*,...},            % if you want to add more keywords to the set
		numbers=left,                    % where to put the line-numbers; possible values are (none, left, right)
		numbersep=5pt,                   % how far the line-numbers are from the code
		numberstyle=\tiny\color{mygray}, % the style that is used for the line-numbers
		rulecolor=\color{black},         % if not set, the frame-color may be changed on line-breaks within not-black text (e.g. comments (green here))
		showspaces=false,                % show spaces everywhere adding particular underscores; it overrides 'showstringspaces'
		showstringspaces=false,          % underline spaces within strings only
		showtabs=false,                  % show tabs within strings adding particular underscores
		stepnumber=2,                    % the step between two line-numbers. If it's 1, each line will be numbered
		stringstyle=\color{mymauve},     % string literal style
		tabsize=2,	                   % sets default tabsize to 2 spaces
		title=\lstname                   % show the filename of files included with \lstinputlisting; also try caption instead of title
	}

	\begin{verbatim}
		\begin{lstlisting}[language=caml]
			let f = fun
			| [] -> 0
			|(t::q) -> 1+(f q)
		\end{lstlisting}
	\end{verbatim}
	
	\begin{lstlisting}[language=caml]
		let f = fun
			| [] -> 0
			|(t::q) -> 1+(f q)
	\end{lstlisting}
	
\end{document}