\documentclass[11pt,a4paper]{article}
\usepackage[utf8]{inputenc}
\usepackage[T1]{fontenc}
\usepackage[francais]{babel}
\usepackage{amsmath}
\usepackage{amsfonts}
\usepackage{amssymb}
\usepackage{graphicx}
\usepackage{fullpage}
\usepackage{tikz}
\usepackage{algorithm}
\usepackage{algpseudocode}
\usepackage{enumitem}

\begin{document}
	\section{Exercice 2}
	\subsection{Algo naif}
	\begin{algorithm}
		\begin{algorithmic}[1]%numéroter toute les n lignes
			\Function{SousSuiteNaif}{$E$}
			\State $somme \gets E[0]$
			\State $d \gets 0$
			\State $f \gets 0$
			\For{$i=0; i<t; n; i++$}
				\For{$j=n-1; j \geq i; i++$}
					\State $p \gets 0$
					\For{$k = i; k \leq j; k++$}
						\State $p += E[k]$
					\EndFor
					\If{$p > somme$}
						\State $somme \gets p$
						\State $d \gets i$
						\State $f \gets j$
					\EndIf
				\EndFor
			\EndFor
			\State \Return $(d, f)$
			\EndFunction
		\end{algorithmic}
	\end{algorithm}

	\subsection{Complexité}
	$C \in \Theta(n^{2})$
	
	\subsection{Chercher à droite}
	
	\begin{algorithm}
		\begin{algorithmic}[1]
			\Function{chercher\_a\_droite}{$E,i,j$}
				\State $s \gets E[i]$
				\State $k \gets i + 1$
				\State $f \gets i$
				\State $smax \gets E[i]$
				\While{$k \leq j$}
					\State $s += E[k]$
					\If{$s > smax$}
						\State $smax \gets s$
						\State $f \gets k$
					\EndIf
				\EndWhile
				\State \Return $(f, smax)$
			\EndFunction
		\end{algorithmic}
	\end{algorithm}
	
	\subsection{Complexité}
	$C \in \Theta(n)$

	\subsection{Spécifier l'algorithme diviser pour régner}
	
	\begin{itemize}
		\item Prend un tableau $E$ en argument
		\item Prend en argument les bornes du tables $i$ et $j$
		\item Algorithme récursif
		\item Renvois le triplet $(s,d,f)$ où :
			\begin{itemize}
				\item $s$ est la somme de la suite $[d, f]$
				\item $d$ est l'index de début de la suite
				\item $f$ est l'index de fin de la suite
			\end{itemize}
	\end{itemize}
	
	\subsection{Expliquer son principe (3 phases)}
	
	

	\begin{center}
		\begin{enumerate}[label=Phase \arabic* :]
			\item Diviser par 2 la liste
			\item Régner sur chaque moitié
			\item Combiner les cas entre
			\begin{itemize}[label={$\to$}]
				\item Maximum de chaque moitié
				\item Les cas de "recollement" dans le cas où la division a "coupé" notre sous-suite de somme maximale (utilisation de $chercher\_a\_droite$ et $chercher\_a\_gauche$)
			\end{itemize}
		\end{enumerate}
	
%		\begin{tikzpicture}[yscale=-1]
%			%\fill (0,0) -- (1,0) -- (1,1) -- cycle;
%			\draw (-4,2) -- (-4,0);
%			\draw (-4,2) -- (4,2);
%			\draw (4,2) -- (4,0);
%			\draw (-4,1.5) rectangle (-2.1,2.5);
%			\draw (-2,1.5) rectangle (2,2.5);
%			\draw (2.1,1.5) rectangle (4,2.5);
%			\draw (0,1) -- (0,3);
%			\draw (-3,1) node {$1$};
%			\draw (0,0) node {$2$};
%			\draw (3,1) node {$3$};
%		\end{tikzpicture}
	\end{center}
	
	\subsection{Écrire l'algorithme}
	
	\begin{algorithm}
		\begin{algorithmic}[1]
			\Function{SousSuite}{$E,i,j$}
				\If{$i == j$} \Comment{Cas d'arrêt}
					\State $smax \gets E[i]$
					\State $dmax \gets i$
					\State $fmax \gets j$
				\Else
					\State $k \gets (i + j) / 2$
				\EndIf
				\State $(smax1, dmax1, fmax1) = Sous\_Suite(E, k+1, j)$ 
				\Comment{Régner}
				\State $(smax2, dmax2, fmax2) = Sous\_Suite(E, i, k)$
				\State $(smax3, dmax3) = chercher\_a\_gauche(E, i, k)$
				\State $(smax4, dmax4) = chercher\_a\_droite(E, k+1, j)$
				\State $(smax3, dmax3, fmax3) \gets (s3 + s4, dmax3, fmax3)$ \Comment{$s3 + s4$ ??}
				\State $(smax4, dmax4, fmax4) \gets (s3 + s4, dmax4, fmax4)$ \Comment{$s3 + s4$ ??}
				\State $smax \gets max(smax1, smax2, smax3, smax4)$ \Comment{Combinaison}
				\If{$smax == smax1$}
					\State \Return $(smax, dmax1, fmax1)$
				\ElsIf{$smax == smax2$}
					\State \Return $(smax, dmax2, fmax2)$
				\ElsIf{$smax == smax3$}
					\State \Return $(smax, dmax3, fmax3)$
				\Else
					\State \Return $(smax, dmax4, fmax4)$
				\EndIf
			\EndFunction
		\end{algorithmic}
	\end{algorithm}
	
	\subsection{Complexité}
	
	??
	
	\subsection{Algorithme à complexité linéaire}
	
	Demandez à Alexandre.
	
\end{document}