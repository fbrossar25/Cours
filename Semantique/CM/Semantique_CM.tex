\documentclass[11pt,a4paper]{article}
\usepackage[utf8]{inputenc}
\usepackage[T1]{fontenc}
\usepackage[francais]{babel}
\usepackage{amsmath}
\usepackage{amsfonts}
\usepackage{amssymb}
\usepackage{graphicx}

\title{Sémantique CM}

\begin{document}
	\maketitle
	
	\section{??}
	$[[...]]$ : pour les instructions\\
	$[[while\ E_{b}\ do\ S ]]y = \lambda e\ si\ [[t_{b}]]_{z} e\ est\ vrai$
	$Alors [[while\ E_{b}\ do\ S]]$

	\section{Sémantique dénotationnelle des boucles}

	Exemple avec une fonction récursive, ATTENTION : on n'est pas dans le langage des tant que.
	On considère $f: \mathbb{Z} \to \mathbb{Z}$\\
	"définie" par $f(x) = si\ x = 0 alors\ 0 sinon\ f(x - 1)$\\
	En fait, $f$ est solution d'une équation de point fixe.\\
	$f = Gf\ avec\ G : (\mathbb{Z} \to \mathbb{Z}) \to (\mathbb{Z} \to \mathbb{Z})$
	(la fin manque)\\
		
	On introduit $\bot_{\mathbb{Z}}$ : indéterminé\\
	$\bot_{\mathbb{Z}} - 1 = \bot_{\mathbb{Z}}$ : le genre de classe\\
	 $\bot_{\mathbb{Z} \to \mathbb{Z}}$ : fonction indéterminée\\
	 $\bot_{\mathbb{Z} \to \mathbb{Z}}(x) = \bot_{\mathbb{Z}}$\\
	 On essaie de voir si certaines informations marchent avec $f_{0} = \bot$\\
	 est-ce que $f = G f_{0}$ ?
	 $$f_{1} = G f_{0} = \lambda x si x = 0 alors\ 0 sinon\ f_{0}(x-1)$$\\
	 $$f_{1} = \lambda x si\ x = 0 alors\ 0 sinon\ \bot \neq f_{0}$$
	 mais $f_{1}$ est plus déterminée que $f_{0}$.
	 $\gamma : E \to E$\\
	 $x = \gamma(x)$ : x point fixe.
	 \begin{align*}
	 	f_{2} &= G f_{1}\\
	 	&= \lambda x . si\ x = 0 alors\ 0\\
	 	&sinon\ si\ x - 1 = 0 alors\ 0\\
	 	&sinon\ \bot\\
	 	f_{2} &= \lambda x . si\ x \in [0,1] alors\ 0\\
	 	&sinon\ \bot\\
	 \end{align*}
	 
	 On continue
	 
	 \begin{align*}
	 	f_{k} &= G^{k} f_{0} = G(f_{k-1})\\
	 	f_{k} &= \lambda x . si\ x \in [0,1] alors\ 0\\
	 	&sinon\ \bot\\
	 \end{align*}
	 En faisant tendre $k \to \infty$ :\\
	 $f_{\infty} = \lambda x . si\ x \in \mathbb{N}\ alors\ 0$\\
	 $sinon\ \bot$
	 
	 Un peu de théorie : introduction de $\mathbb{L}$ littéraux $\mathbb{B}, \mathbb{N}, \mathbb{Z}, \mathbb{Q}$. On ajoute un symbole d'indétermination : $\bot_{\mathbb{B}}, \bot_{\mathbb{N}}, \bot_{\mathbb{Z}}, \bot_{\mathbb{Q}}$.
	 
	 ...
	 
	 
	 
	 On étend cette notion aux fonctions E : ensemble avec $\bot$ et $\sqsubseteq$.\\
	 \begin{align*}
		 f : E \to E\ et\ g : E \to E\\
		 f \sqsubseteq g\\
	 \end{align*}
	 ......
	 
	 propriétés de $\sqsubseteq$ :
	 \begin{itemize}
	 	\item $\sqsubseteq$ est une relation d'ordre :
	 	\begin{align*}
	 		f \sqsubseteq f\\
	 		si\ f \sqsubseteq g\ et\ g \sqsubseteq f\ alors\ f = g\\
	 		si\ f \sqsubseteq g\ et\ g \sqsubseteq h\ alors\ f \sqsubseteq h\\
	 	\end{align*}
	 	\item c'est un ordre partiel
	 	\item toute suite croissante admet une borne supérieure (plus petit majorant)\\
	 	$f_{0} \sqsubseteq f_{1} \sqsubseteq f_{2} ... \sqsubseteq f_{k} \sqsubseteq f_{k+1} ....$
	 \end{itemize}
 	Il existe $f_{\infty}$ une borne sup définie par
 	$$\begin{cases}
	 	Df_{\infty} = \underset{k}{\cup} Df_{k}\\
	 	f_{\infty}(x) = f_{k}(x) pour\ x \in Df_{k}
 	\end{cases}$$
 	Notation : $f_{\infty}$ est notée $\underset{k}{lim}\uparrow f_{k}$ ou $lim\uparrow f_{k}$
 	
 	Définitions:
 	\begin{itemize}
 		\item Soit $E$ et $F$ deux domaines et $f : E \to F$, on dit que f est monotone ssi $\forall x,y \in E, \exists x \sqsubseteq y \implies f(x) \sqsubseteq f(y)$
 		\item si $E$ est un domaine, et $f : E \to E$, on dit que $f$ est continue ssi
 		\begin{enumerate}
	 		\item $f$ est monotone
	 		\item ta mère la catin
 		\end{enumerate}
 	\end{itemize}
\end{document}