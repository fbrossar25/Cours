\documentclass[11pt,a4paper]{article}
\usepackage[utf8]{inputenc}
\usepackage[T1]{fontenc}
\usepackage[francais]{babel}
\usepackage[fleqn]{amsmath}
\usepackage{amsfonts}
\usepackage{amssymb}
\usepackage{stmaryrd}
\usepackage{graphicx}

\title{Sémantique TD3}

\newcommand{\ra}{\rightarrow}

\begin{document}
	\section*{Sémantique dénotationnelle}
	\subsection*{Environnements}
	\begin{tabular}{|c|c|}
		\hline 
		Nom des variables & Valeur \\ 
		\hline 
		x & 4 \\ 
		\hline 
		y & 5 \\ 
		\hline 
	\end{tabular}
	\begin{align*}
		\sigma ,{\sigma }^{\prime }\ \ \ \sigma \left[x{\to }4\right]\\
		[[x*x - 4*y+z]](\sigma) &= [[x*x]](\sigma) - [[4*y*z]](\sigma)\\
		\text{liste de variables x y z}&\\
		&= [[x]](\sigma) * [[x]](\sigma) - [[4]]\sigma*[[y]]\sigma*[[z]]\sigma\\
		&= \sigma(x) * \sigma(x) - 4 * \sigma(y) * \sigma(z)\\
	\end{align*}
	\begin{align*}
		x,y,z &\to &t,u,v,w\\
		\sigma &\to &\sigma &\ \ \ \text{approche impérative}\\
		\sigma &\to &int * int * int * int &\ \ \ \text{approche fonctionnelle}
		\text{il à tout effacé ce con !}
	\end{align*}
	Soit $e$ une expression :
	\begin{description}
		\item[aff] $[[x := e]](\sigma) = \sigma[x \to [[e]](\sigma)]$
		\item[seq]  $[[I, J]](J) = [[J]] o [[I]](\sigma) = [[J]]([[I]](\sigma))$
		\item[condition] $[[if\ b then\ i1 else\ i2]](\sigma) =$
		$\begin{cases}
			[[i1]](\sigma) & si\ [[b]](\sigma) = true\\
			[[i2]](\sigma) & sinon\\
		\end{cases}$
		\item[while] $[[while\ b do\ S]](\sigma) =$
		$\begin{cases}
			[[S; while\ b do\ S]](\sigma) si\ [[b]](\sigma) = true\\
			\sigma sinon\\
		\end{cases}$
	\end{description}
	\newpage
	\section{Expressions, instructions}
	\subsection{Question 2}
	\begin{align*}
		[[tmp := x; x := y; ]]\\
		\text{demander a dedele}\\
	\end{align*}
	\subsection{Question bonus !!}
	1) Écrire l'échange de 2 variables entières sans utiliser de variables intermédiaires, uniquement avec des additions et des soustractions.
	
	2) Vérifier en calculant la sémantique que c'est bien un échange.
	\\\\\\
	2 variable x et y\ \ \   $x_{0}, y_{0}$
	\begin{align*}
		x &:= x + y;\\
		y &:= x - y;\\
		x &:= x - y;\\
	\end{align*}
	$\sigma(x) = x_{0} \\ \sigma(y) = y_{0}$
	\begin{align*}
		[[x := x + y; y := x - y; x := x - y]](\sigma) &= [[y := x - y; x := x - y]]([[x := x + y]](\sigma))\\
		&= [[y := x - y; x := x - y]](\sigma[x \to [[x + y]](\sigma)])\\
		&= [[y := x - y; x := x - y]](\sigma[x \to x_{0} + y_{0}])\\
		&= [[x := x - y]]([[y := x - y]](\sigma[x \to x_{0} + y_{0}]))\\
		&= [[x := x - y]](\sigma[y \to x_{0} + y_{0} - y_{0}; x \to x_{0} + y_{0}])\\
	\end{align*}
	3) \underline{Variante} : on peut faire la même chose avec la multiplication et la division
	\newpage
	
	\subsection{Question 5}
	Rappel :\\
	\[
		[[if\ b\ then\ i1\ else\ i2]](\sigma) =
		\begin{cases}
			[[i1]](\sigma) si\ [[b]](\sigma) = true\\
			[[i2]](\sigma) sinon\\
		\end{cases}
	\]
	
	\begin{align*}
		[[if\ x1 > x2\ then\ m := x1 else\ m := x2]](\sigma) &=
		\begin{cases}
			[[m := x1]](\sigma)\ si\ [[x1 > x2]](\sigma) = true\\
			[[m := x2]](\sigma)\ sinon\\
		\end{cases}\\
		&=
		\begin{cases}
			\sigma[m \to \sigma(x1)]\ si\ \sigma(x1) > \sigma(x2)\\
			\sigma[m \to \sigma(x2)]\ sinon\\
		\end{cases}\\
		&= \sigma[m \to max(\sigma(x1), \sigma(x2))]\\
	\end{align*}
	\begin{align*}
		max : int \to int \to int&\\
		[[max(x,y)]](\sigma) &=
		\begin{cases}
			\sigma(x)\ si\ \sigma(x) > \sigma(y)\\
			\sigma(y)\ sinon\\
		\end{cases}
	\end{align*}
\end{document}